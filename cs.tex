\documentclass[leqno]{scrartcl}
\usepackage[a3paper,landscape, left=1cm,right=1cm, top=0cm, bottom=1cm,includeheadfoot]{geometry}
\usepackage[utf8]{inputenc}   % Zeichenkodierung: UTF-8 (für Umlaute)   
\usepackage[german]{babel}    % Deutsche Sprache
\usepackage{multicol}     % Spaltenpaket
\usepackage{amsmath}
\usepackage{amssymb}
\usepackage{multicol}     % ermöglicht Seitenspalten  
\usepackage{wasysym}      % Blitz
\usepackage{graphicx}
\usepackage{xcolor}
\KOMAoption{fontsize}{12pt}
\usepackage{blindtext}
\usepackage{tikz}
\usepackage{enumitem}
\setitemize{noitemsep,topsep=0pt,parsep=0pt,partopsep=0pt}

% \setlength{\abovedisplayskip}{0pt}
% \setlength{\belowdisplayskip}{0pt}
% \setlength{\abovedisplayshortskip}{0pt}
% \setlength{\belowdisplayshortskip}{0pt} 
\makeatletter
\g@addto@macro\normalsize{%
  \setlength\abovedisplayskip{5pt}
  \setlength\belowdisplayskip{5pt}
  \setlength\abovedisplayshortskip{5pt}
  \setlength\belowdisplayshortskip{5pt}
}


\begin{document}
\begin{multicols}{3}

\section{Basics (Non-FM)}

\textbf{Erwartungswert:}
  \begin{equation}
    E[X]=\mu=\sum_{i \in I} p_i x_i
  \end{equation}
\textbf{Varianz:}
  \begin{equation*}
    \begin{split}
      Var[X]&=\sigma^2\\
          &=E[(X-E[X])^2]\\
          &=E[X^2]-E[X]^2\\
    \end{split}
  \end{equation*}
\textbf{Kovarianz:}
  \begin{equation}
    \begin{split}
    Cov[X,Y]&=E[(X-E[X])(Y-E[Y])]\\
    Cov[X,X]&=Var[X]
    \end{split}
  \end{equation}
\textbf{Korrelationskoeffizient:}
  \begin{equation*}
    \begin{split}
      Corr[X,Y] &= \varrho [X,Y]\\
                &= \frac{Cov[X,Y]}{\sigma (X) \sigma (Y)}\\
                &= \frac{Cov[X,Y]}{\sqrt{Var[X]}\sqrt{Var[Y]}}\\
    \end{split}
  \end{equation*}
\textbf{p,q Formel:}
  \begin{equation}
    x_{1,2}=\frac{-b \pm \sqrt{b^2 - 4ac}}{2a}
  \end{equation}

\section{Entscheidungstheorie}

\subsection{Dominanzkonzept}

\begin{equation*}
\begin{split}
        \text{A dominiert B strikt} \Leftrightarrow & \forall t \in T : e_{A,t} > e_{B,t}\\
        \text{A dominiert B (schwach)} \Leftrightarrow & \forall t \in T : e_{A,t} \geq e_{B,t} \wedge \exists t \in T : e_{A,t} > e_{B,t}\\
\end{split}
\end{equation*}
Das heißt, dass A zu allen Zeitpunkten größer oder gleich und mindestens ein mal 
echt größer als B ist. Ist eine Alternative nicht dominiert, dann ist sie 
effizient. Gibt es nur eine effiziente, ist es die dominante Alternative.

\subsection{Erwartungsnutzentheorie}

\textbf{Eigenschaft eines Marktteilnehmers:}
  \begin{equation*}
    \begin{split}
      \text{Risikoaversion}    &\Leftrightarrow E[U(X)] < U(E[X]) \\
      \text{Risikoneutralität} &\Leftrightarrow E[U(X)] = U(E[X]) \\
      \text{Risikofreude}      &\Leftrightarrow E[U(X)] > U(E[X]) \\
    \end{split}
  \end{equation*}
\textbf{Sicherheitsäquivalent (CE):}
  \begin{equation}
    \begin{split}
      U(CE) &= E[U(X)]\\
        CE &= U^{-1}(E[U(X)])\\
    \end{split}
  \end{equation}
\textbf{Risikoprämie (RP):}
  \begin{equation}
    RP = E[X] - CE
  \end{equation}

\subsection{Erwartungswert-Varianz-Prinzip}

\textbf{Eigenschaft eines Marktteilnehmers:}
  \begin{equation*}
    \begin{split}
      \text{Risikoaversion}    &\Leftrightarrow \frac{\delta U}{\delta \sigma} < 0 \\
      \text{Risikoneutralität} &\Leftrightarrow \frac{\delta U}{\delta \sigma} = 0 \\
      \text{Risikofreude}      &\Leftrightarrow \frac{\delta U}{\delta \sigma} > 0 \\
    \end{split}
  \end{equation*}
\textbf{Sicherheitsäquivalent} (CE):
  \begin{equation}
    \varphi(CE, 0) = \varphi(\mu, \sigma)
  \end{equation}
\textbf{Risikoprämie (RP):}
  \begin{equation}
    RP = E[X] - CE
  \end{equation}
\textbf{Merke:} ($\mu$,$\sigma$)-Prinzip steht nur bei quadratischen und
exponientiellen Nutzenfunktionen im Einklang mit dem Bernulliding

\section{Investitionsrechnung}

\subsection{Finanzmathematik}

\textbf{Endwert} (aufzinsen)(terminal value):
  \begin{equation}
    \begin{split}
      EW &= z_1(1+i)^{T-1} + \dots + z_{T-1}(1+i) + Z_T \\
        &= \sum^T_{t=1} z_t (1+i)^{T-t}\\
    \end{split}
  \end{equation}
\textbf{Barwert} (abzinsen)(present value, Gegenwartswert):
  \begin{equation}
    \begin{split}
      BW &= \frac{z_1}{1+i}+\frac{z_2}{(1+i)^2}+\dots+\frac{z_T}{(1+i)^T}\\
          &= \sum^T_{t=\color{red}{\textbf{1}}}\frac{z_t}{(1+i)^t}\\
    \end{split}
  \end{equation}
\textbf{Kapitalwert} (abzinsen)(net present value‚ Nettogegenwartswert, Nettobarwert)
  \begin{equation}
    KW = \sum^T_{t=\color{red}{\textbf{0}}}\frac{z_t}{(1+i)^t}
  \end{equation}
\textbf{Merke:} Bei PV wird die Anfangsauszahlung nicht abgezogen, beim NPV schon.

\subsection{Investitionsrechnung}

\textbf{Konvergenz geometrischer Reihen:}
  \begin{equation}\label{convergence}
    \begin{split}
    |\delta|<1 \rightarrow \sum^T_{t=0}\delta^t=\frac{1}{1-\delta}\\
    \implies \forall i > 0 : \sum^T_{t=0}\left(\frac{1}{1+i}\right)^t  = \frac{1+i}{i}\\
    \end{split}
  \end{equation}
\textbf{Ewige Rente (Zahlungsbeginn k):}
  \begin{equation}
    KW = \sum^{\infty}_{t=\color{red}{\textbf{k}}}\frac{A}{(1+i)^t} \overset{(\ref{convergence})}{=}\frac{A}{i}(1+i)^{\color{red}{\textbf{1-k}}}
  \end{equation}
\textbf{Ewige, nachschüssige Rente:}
  \begin{equation}
    KW = \sum^{\infty}_{t=\color{red}{\textbf{1}}}\frac{A}{(1+i)^t}=\frac{A}{i}
  \end{equation}
\textbf{Ewige, vorschüssige  Rente:}
  \begin{equation}
    KW = \sum^{\infty}_{t=\color{red}{\textbf{0}}}\frac{A}{(1+i)^t}=\frac{A}{i}(1+i)^{\color{lightgray}{\textbf{1}}}
  \end{equation}
\textbf{KW $\sim$ EW:}
  \begin{equation}\label{relation}
    \begin{split}
    &EW = KW (1+i)^{T}\\
    \iff &KW = EW (1+i)^{-T} \\
    \end{split}
  \end{equation}
\textbf{Annuität}:\\
  Eine zur Zahlungsreihe $z_t$ ($t$ = 0, 1, ..., T) äquivalente Annuität ist eine Reihe 
  gleich hoher Zahlungen A zu den Zeitpunkten 1, 2, ..., T, deren Kapitalwert 
  gleich dem der ursprünglichen Zahlungsreihe ist.
  \begin{equation*}
    \begin{split}
      KW &= \sum^T_{t=1}\frac{A}{(1+i)^t}\\
      KW &= \color{cyan}\underbrace{\frac{A}{i}}_\text{Ewige Rente (k=1)}
            \color{black}-
            \color{magenta}\underbrace{\frac{A}{i}\cdot\frac{1}{(1+i)^T}}_\text{Ewige Rente (k=T)}\\
      KW &= A\color{orange} \underbrace{\left(\frac{1}{i}-\frac{1}{i(1+i)^T}\right)}_\text{Rentenbarwertfaktor}\\
      KW &= A\color{orange} \underbrace{\left(\frac{(1+i)^T-1}{(1+i)^T \cdot i}\right)}_\text{Rentenbarwertfaktor}\\
      A  &= KW \underbrace{\left(\frac{(1+i)^T \cdot i}{(1+i)^T-1}\right)}_\text{Annuitätenfaktor}\\
    \end{split}
  \end{equation*}
\begin{center}
  \begin{tikzpicture}
    \draw [|-latex,line width=1pt, cyan] (0,1.0) -- node[above] {1, 2, 3, \dots} (8,1.0);
    \draw [|-latex,line width=1pt, magenta] (5,0.5) -- node[above] {T+1, \dots} (8,0.5);
    \draw [|-|,line width=1pt, orange] (0,0.0) -- node[above] {1, 2, 3, \dots, T} (5,0.0);
  \end{tikzpicture}
\end{center}
\textbf{AF $\sim$ RBF:}
  \begin{equation}
    AF = RBF^{-1}
  \end{equation}
\textbf{RBF$_n$ $\sim$ RBF$_v$:}
  \begin{equation}\label{RBFsim}
    RBF_v = RBF_n \cdot (1+i)
  \end{equation}
Nach(Vor)schüssig bedeutet, dass Zahlungen direkt nach(vor) dem Zins 
stattfinden. D.h. für den EW, dass ein mal weniger(mehr) verzinst wird und für 
den BW, dass ein mal mehr(weniger) abgezinst wird.\\
\textbf{Rentenbarwert (nachschüssig):}
  \begin{equation}
    BW_n = A \cdot RBF_n = A \cdot \left(\frac{(1+i)^T - 1}{(1+i)^T \cdot i}\right)
  \end{equation}
\textbf{Rentenbarwert (vorschüssig):}
  \begin{equation}
    BW_v = A \cdot RBF_v \overset{\ref{RBFsim}}{=} A \cdot \left(\frac{(1+i)^T - 1}{(1+i)^T \cdot i}\right)\color{red} \cdot (1+i) 
  \end{equation}
\textbf{Rentenendwert (nachschüssig):}
  \begin{equation}
    EW_n = A \cdot REF_n = A \cdot \left(\frac{(1+i)^T - 1}{i}\right)
  \end{equation}
\textbf{Rentenendwert (vorschüssig):}
  \begin{equation}
    EW_v = A \cdot REF_v \overset{\ref{RBFsim}}{=} A \cdot \left(\frac{(1+i)^T - 1}{i}\right)\color{red} \cdot (1+i)
  \end{equation}
\textbf{Interner Zins:}
  Es sei ($z_0, z_1, \dots, z_T$) ein Zahlungsstrom. Der interne Zins ist eine Zahl $i^*$ , die 
  die folgende Gleichung löst:
  $$0 = z_0 + \frac{z_1}{(1+i^*)^1} + \cdots + \frac{z_T}{(1+i^*)^T} = \sum^T_{t=0}\frac{z_t}{(1+i)^t}$$
  Ersetze dazu $\delta^* = \frac{1}{(1+i)^t}$ und erreche $i^* = \frac{1}{\delta^*}-1$

\subsection{Preinreich-Lücke-Theorem}

\textbf{Kongruenzprinzip:}
  \begin{equation}\label{Kongruenzprinzip}
    \sum^{T}_{t=0} (e_t-a_t) = \sum^{T}_{t=0} (L_t-K_t)
  \end{equation}
\textbf{Residualgewinn:}
  \begin{equation}\label{Residualgewinn}
    RG_t=\underbrace{L_t-K_t}_\text{Periodengewinn}-\underbrace{i \cdot KB_{t-1}}_\text{Kapitalbindungszinsen}
  \end{equation}
\textbf{Preinreich-Lücke-Theorem:}
  \begin{equation}
    \begin{split}
      \sum^{T}_{t=0} \frac{RG_t}{(1+i)^t} 
        &\overset{\ref{Residualgewinn}}{=} \sum^{T}_{t=0} \frac{L_t - K_t - i \cdot KB_{t-1}}{(1+i)^t}\\
        &\overset{\ref{Kongruenzprinzip}}{\equiv} \sum^{T}_{t=0} \frac{e_t - a_t}{(1+i)^t}
        = \sum^{T}_{t=0} \frac{cf_t}{(1+i)^t}
        =: KW
    \end{split}
  \end{equation}

\section{Capital Asset Pricing Model (CAPM)}

\subsection{Bewertung des Risikos}
  \textbf{Das Beta – Marktrisikosensitivität:}
    \begin{equation}
      \beta_i = \frac{Cov(r_i ,r_M)}{Var(r_M)}
    \end{equation}
  \textbf{Marktrisikoprämie:}
    \begin{equation}
      MRP = \mu(r_M) -r_f
    \end{equation}
  \textbf{Risikoprämie für ein WP$_i$:}
    \begin{equation}
      RP_i = \beta_i \cdot MRP
    \end{equation}

\subsection{Optimale Portfolioallokation und CAPM}

\textbf{Erwartete Rendite eines Portfolios P aus n WPs:}
  \begin{equation}
    E[r_P] =\mu_P= \sum^{n}_{i=1} a_i \cdot E[r_i]
  \end{equation}
\textbf{Varianz eines Portfolios P aus n WPs:}
  \begin{equation}
    \begin{split}
      Var[r_P]&=\sigma_P^2\\
          &=E[(r_p-E[r_p])^2]\\
          &=E[r_p^2]-E[r_p]^2\\
          &=\sum^{n}_{i=1}\sum^{n}_{j=1}a_i \cdot a_j \cdot Cov[r_i,r_j]\\
          &\overset{\color{red}{n=2}}{=} a^2_1 \sigma^2_1 + 2a_1a_2 \cdot Cov[r_1,r_2] + a^2_2 \sigma^2_2
    \end{split}
  \end{equation}
\textbf{Kovarianz zweier Wertpapiere:}
  \begin{equation}
      Cov[r_i,r_j] = E[(r_i-E[r_i]) \cdot (r_j-E[r_j])]
  \end{equation}
\textbf{Korrelation zweier Wertpapiere:}
  \begin{equation}
    \begin{split}
      Corr[r_i,r_j] &= \varrho [r_i,r_j] \in [-1,1]\\
                    &= \frac{Cov[r_i,r_j]}{\sigma (r_i) \cdot \sigma (r_j)}\\
    \end{split}
  \end{equation}
\textbf{Bedingungen für effiziente Portfolios:}
  \begin{enumerate}
    \item Portfolio minimiert bei gegebener erwarteter Rendite die Varianz
    \item Portfolio wird nicht dominiert
  \end{enumerate}
\subsubsection{Kombination von $P$ mit risikoloser Anlage $f$:}
Investor mischt $P$ mit einem sicheren Wertpapier $f$ mit der Rendite $r_f$ ($P(b)$). Anteil von $f$ betrage $(1 - b)$.\\
\textbf{Erwartete Rendite:}
  \begin{equation}
    \begin{split}
      E[r_{P(b)}]&=(1-b) \cdot r_f + b \cdot E[r_P]\\
                 &=r_f + b \cdot (E[r_P ] - r_f)
    \end{split}
  \end{equation}
\textbf{Varianz:}
  \begin{equation}
    \begin{split}
      Var[r_{P(b)}] &= (1-b)^2 \cdot \overbrace{Var[r_f]}^{0}+ b^2 \cdot Var[r_P]\\
                  &+ 2 \cdot (1-b) \cdot b \cdot \underbrace{Cov[r_f ,r_P ]}_{0}\\
                  &= b^2 \cdot Var[r_P]
    \end{split}
  \end{equation}
\subsubsection{CAPM:}
\textbf{Annahmen:}
  \begin{itemize}
    \item Anleger sind rational und risikoavers
    \item Der Markt ist vollkommen und friktionsfrei
    \begin{itemize}
      \item[$\bullet$] Homogene Erwartungen bezüglich erwarteter Rendite,\\ Volatilität und Korrelationen 
      \item[$\bullet$] Kauf- und Verkaufspreise sind identisch
      \item[$\bullet$] Keine Transaktionskosten
      \item[$\bullet$] Kurze Positionen (short positions) zulässig
      \item[$\bullet$] Keine Arbitragemöglichkeiten
    \end{itemize}
  \end{itemize} 
\textbf{Kapitalmarktlinie (Zweipunkteform mit $x_1=0$):}
  \begin{equation}
    \mu_{KML} = r_f + \frac{\mu_M - r_f}{\sigma_M}\cdot \sigma_{KML}
  \end{equation}
\textbf{Capital-Asset-Pricing-Modell:}
  \begin{equation}
    \begin{split}
      E[r_i] &= r_f +  \underbrace{\underbrace{(E[r_M] - r_f)}_{MRP} \cdot \underbrace{\frac{Cov[r_i,r_M]}{Var[r_M]}}_{\beta_i}}_{RP}\\
    \end{split}
  \end{equation}

\subsection{Schätzung der Kapitalkosten}

  \begin{equation}
  \end{equation}
  \begin{equation}
  \end{equation}

\section{Unternehmens- und Projektbewertung}

\subsection{DCF-Verfahren}
\subsection{Unternehmensbewertung mit Steuern}
\subsection{Wertermittlung mit Gewinnen}

\section{Performancemaße}

\subsection{Bemessungsgrundlagen für Bonussysteme}
\subsection{Beispiele zur Anreizwirkung}
\subsection{Verfeinerung des Residualgewinnkonzeptes}

\section{Investitionssteuerung}

\subsection{Residualgewinnorientierte INvestitionssteuerung}
\subsection{Periodische Performancemaße}
\subsection{Bonusbanken}
\subsection{Implikationen für die Unternehmenssteuerung}

\end{multicols}
\end{document}


