\documentclass[a4paper]{scrartcl}
\usepackage[landscape, left=1cm,right=1cm, top=0cm, bottom=1cm,includeheadfoot]{geometry}
\usepackage[utf8]{inputenc}   % Zeichenkodierung: UTF-8 (für Umlaute)   
\usepackage[german]{babel}    % Deutsche Sprache
\usepackage{multicol}     % Spaltenpaket
\usepackage{amsmath}
\usepackage{amssymb}
\usepackage{multicol}     % ermöglicht Seitenspalten  
\usepackage{wasysym}      % Blitz
\usepackage{graphicx}
\usepackage{xcolor}
\KOMAoption{fontsize}{10pt}
\usepackage{blindtext}
\usepackage{tikz} 


\begin{document}
\begin{multicols}{3}

\section{Basics (Non-FM)}

\textbf{Erwartungswert:}
$$E[X]=\mu=\sum_{i \in I} p_i x_i $$
\textbf{Varianz:}
\begin{equation*}
\begin{split}
Var[X]&=\sigma^2\\
      &=E[(X-E[X])^2]\\
      &=E[X^2]-E[X]^2\\
\end{split}
\end{equation*}
\textbf{Kovarianz:}
$$Cov[X,Y]=E[(X-E[X])(Y-E[Y])]$$
\textbf{Korrelationskoeffizient:}
\begin{equation*}
\begin{split}
    Corr[X,Y] &= \varrho [X,Y]\\
              &= \frac{Cov[X,Y]}{\sigma (X) \sigma (Y)}\\
              &= \frac{Cov[X,Y]}{\sqrt{Var[X]}\sqrt{Var[Y]}}\\
\end{split}
\end{equation*}
\textbf{p,q Formel:}
$$x_{1,2}=\frac{-b \pm \sqrt{b^2 - 4ac}}{2a}$$

\section{Entscheidungstheorie}
\subsection{Dominanzkonzept}

\begin{equation*}
\begin{split}
       & \text{A dominiert B strikt} \\
     \Leftrightarrow & \forall t \in T : e_{A,t} > e_{B,t}\\
       & \text{A dominiert B (schwach)} \\
     \Leftrightarrow & \forall t \in T : e_{A,t} \geq e_{B,t} \wedge \exists t \in T : e_{A,t} > e_{B,t}\\
\end{split}
\end{equation*}
Das heißt, dass A zu allen Zeitpunkten größer oder gleich und mindestens ein mal echt größer als B ist.\\
Ist eine Alternative nicht dominiert, dann ist sie effizient. Gibt es nur eine effiziente, ist es die dominante Alternative.

\subsection{Erwartungsnutzentheorie}

\textbf{Eigenschaft eines Marktteilnehmers:}
\begin{equation*}
\begin{split}
     \text{Risikoaversion}    &\Leftrightarrow E[U(X)] < U(E[X]) \\
     \text{Risikoneutralität} &\Leftrightarrow E[U(X)] = U(E[X]) \\
     \text{Risikofreude}      &\Leftrightarrow E[U(X)] > U(E[X]) \\
\end{split}
\end{equation*}
\textbf{Sicherheitsäquivalent (CE):}
\begin{equation*}
\begin{split}
U(CE) &= E[U(X)]\\
   CE &= U^{-1}(E[U(X)])\\
\end{split}
\end{equation*}
\textbf{Risikoprämie (RP):}
$$RP = E[X] - CE$$

\subsection{Erwartungswert-Varianz-Prinzip}

\textbf{Sicherheitsäquivalent} (CE):
$$\varphi(CE, 0) = \varphi(\mu, \sigma) $$
\textbf{Risikoprämie (RP):}
$$RP = E[X] - CE$$
\textbf{Merke:} ($\mu$,$\sigma$)-Prinzip steht nur bei quadratischen und
exponientiellen Nutzenfunktionen im Einklang mit dem Bernullidingens

\section{Investitionsrechnung}

\subsection{Finanzmathematik}

\textbf{Endwert} (aufzinsen)(terminal value):
\begin{equation*}
\begin{split}
EW &= z_1(1+i)^{T-1} + \dots + z_{T-1}(1+i) + Z_T \\
   &= \sum^T_{t=1} z_t (1+i)^{T-t}\\
\end{split}
\end{equation*}
\textbf{Barwert} (abzinsen)(present value, Gegenwartswert):
\begin{equation*}
\begin{split}
BW &= \frac{z_1}{1+i}+\frac{z_2}{(1+i)^2}+\dots+\frac{z_T}{(1+i)^T}\\
     &= \sum^T_{t=\color{red}{\textbf{1}}}\frac{z_t}{(1+i)^t}\\
\end{split}
\end{equation*}
\textbf{Kapitalwert} (abzinsen)(net present value‚ Nettogegenwartswert, Nettobarwert)
$$KW = \sum^T_{t=\color{red}{\textbf{0}}}\frac{z_t}{(1+i)^t}$$
\textbf{Merke:} Bei PV wird die Anfangsauszahlung nicht abgezogen, beim NPV schon.

\subsection{Investitionsrechnung}

\color{gray}
Konvergenz geometrischer Reihen:
$$|\delta|<1 \rightarrow \sum^T_{t=0}\delta^t=\frac{1}{1-\delta}$$
$$ \Rightarrow i > 0 \rightarrow \iff \implies \sum^T_{t=0}\left(\frac{1}{1+i}\right)^t  = \frac{1+i}{i}$$
\color{black}
\textbf{Ewige Rente (Zahlungsbeginn k):}
$$KW = \sum^{\infty}_{t=\color{red}{\textbf{k}}}\frac{A}{(1+i)^t}=\frac{A}{i}(1+i)^{\color{red}{\textbf{1-k}}}$$
\textbf{Ewige, nachschüssige Rente:}
$$KW = \sum^{\infty}_{t=\color{red}{\textbf{1}}}\frac{A}{(1+i)^t}=\frac{A}{i}$$
\textbf{Ewige, vorschüssige  Rente:}
$$KW = \sum^{\infty}_{t=\color{red}{\textbf{0}}}\frac{A}{(1+i)^t}=\frac{A}{i}(1+i)^{\color{lightgray}{\textbf{1}}}$$
\textbf{KW $\sim$ EW:}
\begin{equation*}
\begin{split}
&EW = KW (1+i)^{T}\\
\iff &KW = EW (1+i)^{-T} \\
\end{split}
\end{equation*}
\textbf{Annuität}:\\
Eine zur Zahlungsreihe $z_t$ ($t$ = 0, 1, ..., T) äquivalente Annuität ist eine Reihe 
gleich hoher Zahlungen A zu den Zeitpunkten 1, 2, ..., T, deren Kapitalwert 
gleich dem der ursprünglichen Zahlungsreihe ist.
\begin{equation*}
\begin{split}
KW &= \sum^T_{t=1}\frac{A}{(1+i)^t}\\
KW &= \color{cyan}\underbrace{\frac{A}{i}}_\text{Ewige Rente (k=1)}
      \color{black}-
      \color{magenta}\underbrace{\frac{A}{i}\cdot\frac{1}{(1+i)^T}}_\text{Ewige Rente (k=T)}\\
KW &= \color{orange}A \underbrace{\left(\frac{1}{i}-\frac{1}{i(1+i)^T}\right)}_\text{Rentenbarwertfaktor}\\
A  &= KW \underbrace{\left(\frac{i(1+i)^T}{(1+i)^T-1}\right)}_\text{Annuitätenfaktor}\\
\end{split}
\end{equation*}
\begin{center}
  \begin{tikzpicture}
    \draw [|-latex,line width=1pt, cyan] (0,1.0) -- node[above] {1, 2, 3, \dots} (8,1.0);
    \draw [|-latex,line width=1pt, magenta] (5,0.5) -- node[above] {T+1, \dots} (8,0.5);
    \draw [|-|,line width=1pt, orange] (0,0.0) -- node[above] {1, 2, 3, \dots, T} (5,0.0);
  \end{tikzpicture}
\end{center}
\textbf{Interner Zins:}

\subsection{Preinreich-Lücke-Theorem}

\textbf{Kongruenzprinzip (1):}
$$\sum^{T}_{t=0} (e_t-a_t) = \sum^{T}_{t=0} (L_t-K_t)$$
\textbf{Residualgewinn (2):}
$$RG_t=\underbrace{L_t-K_t}_\text{Periodengewinn}-\underbrace{i \cdot KB_{t-1}}_\text{Kapitalbindungszinsen}$$
\textbf{Preinreich-Lücke-Theorem:}
\begin{equation*}
\begin{split}
\sum^{T}_{t=0} \frac{RG_t}{(1+i)^t} 
  &\overset{(2)}{=} \sum^{T}_{t=0} \frac{L_t - K_t - i \cdot KB_{t-1}}{(1+i)^t}\\
  &\overset{(1)}{\equiv} \sum^{T}_{t=0} \frac{e_t - a_t}{(1+i)^t}
  = \sum^{T}_{t=0} \frac{cf_t}{(1+i)^t}
  =: KW
\end{split}
\end{equation*}

\section{Capital Asset Pricing Model (CAPM)}

\blindtext

\section{Unternehmens- und Projektbewertung}

\blindtext

\section{Performancemaße}

\blindtext

\section{Investitionssteuerung}

\blindtext
\blindtext
\blindtext
\blindtext
\blindtext
\blindtext
\blindtext
\blindtext
\blindtext
\blindtext
\blindtext
\blindtext
\blindtext

\end{multicols}
\end{document}


