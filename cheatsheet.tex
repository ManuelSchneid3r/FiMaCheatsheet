\documentclass[leqno]{scrartcl}
\usepackage[a3paper, landscape,left=1cm,right=1cm, top=0cm, bottom=1cm,includeheadfoot]{geometry}
\usepackage[utf8]{inputenc}   % Zeichenkodierung: UTF-8 (für Umlaute)   
\usepackage[german]{babel}    % Deutsche Sprache
\usepackage{multicol}     % Spaltenpaket
\usepackage{amsmath}
\usepackage{amssymb}
\usepackage{multicol}     % ermöglicht Seitenspalten  
\usepackage{wasysym}      % Blitz
\usepackage{graphicx}
\usepackage{xcolor}
\KOMAoption{fontsize}{12pt}
\usepackage{blindtext}
\usepackage{tikz}
\usepackage{enumitem}
\setitemize{noitemsep,topsep=0pt,parsep=0pt,partopsep=0pt}

% \setlength{\abovedisplayskip}{0pt}
% \setlength{\belowdisplayskip}{0pt}
% \setlength{\abovedisplayshortskip}{0pt}
% \setlength{\belowdisplayshortskip}{0pt} 
\makeatletter
\g@addto@macro\normalsize{%
  \setlength\abovedisplayskip{5pt}
  \setlength\belowdisplayskip{5pt}
  \setlength\abovedisplayshortskip{5pt}
  \setlength\belowdisplayshortskip{5pt}
}


\begin{document}
\begin{multicols}{4}

%-------------------------------------------------------------------------------

\section{Basics (Non-FM)}

\textbf{Erwartungswert:}
  \begin{equation*}
    E[X]=\mu=\sum_{i \in I} p_i x_i
  \end{equation*}
\textbf{Varianz:}
  \begin{equation*}
    \begin{split}
      Var[X]&=\sigma^2\\
      Var[X]&=E[(X-E[X])^2]\\
      Var[X]&=E[X^2]-E[X]^2\\
    \end{split}
  \end{equation*}
\textbf{Kovarianz:}
  \begin{equation*}
    \begin{split}
    Cov[X,Y]&=E[(X-E[X])(Y-E[Y])]\\
    Cov[X,X]&=Var[X]
    \end{split}
  \end{equation*}
\textbf{Korrelationskoeffizient:}
  \begin{equation*}
    \begin{split}
      Corr[X,Y] &= \varrho [X,Y]\\
      Corr[X,Y] &= \frac{Cov[X,Y]}{\sigma (X) \sigma (Y)}\\
    \end{split}
  \end{equation*}
\textbf{a-b-c Formel:}
  \begin{equation*}
    x_{1,2}=\frac{-b \pm \sqrt{b^2 - 4ac}}{2a}
  \end{equation*}
\textbf{p-q Formel:}
  \begin{equation*}
    x_{1,2}=-\frac{p}{2}\pm\sqrt{\left(\frac{p}{2}\right)^2-q}
  \end{equation*}

%-------------------------------------------------------------------------------

\section{Entscheidungstheorie}

\subsection{Dominanzkonzept}
\begin{align*}
&\text{A dominiert B strikt}\\
\Leftrightarrow  &\forall t \in T : e_{A,t} > e_{B,t}\\
&\text{A dominiert B (schwach)}\\
\Leftrightarrow  &\forall t \in T : e_{A,t} \geq e_{B,t} \wedge \exists t \in T : e_{A,t} > e_{B,t}\\
\end{align*}
Das heißt, dass A zu allen Zeitpunkten größer oder gleich und mindestens ein mal 
echt größer als B ist. Ist eine Alternative nicht dominiert, dann ist sie 
effizient. Gibt es nur eine effiziente, ist es die dominante Alternative.

\subsection{Erwartungsnutzentheorie}

\textbf{Eigenschaft eines Marktteilnehmers:}
  \begin{equation*}
    \begin{split}
      \text{Risikoaversion}    &\Leftrightarrow E[U(X)] < U(E[X]) \\
      \text{Risikoneutralität} &\Leftrightarrow E[U(X)] = U(E[X]) \\
      \text{Risikofreude}      &\Leftrightarrow E[U(X)] > U(E[X]) \\
    \end{split}
  \end{equation*}
\textbf{Sicherheitsäquivalent (CE):}
  \begin{equation*}
    \begin{split}
      U(CE) &= E[U(X)]\\
        CE &= U^{-1}(E[U(X)])\\
    \end{split}
  \end{equation*}
\textbf{Risikoprämie (RP):}
  \begin{equation*}
    RP = E[X] - CE
  \end{equation*}

\subsection{Erwartungswert-Varianz-Prinzip}

\textbf{Eigenschaft eines Marktteilnehmers:}
  \begin{equation*}
    \begin{split}
      \text{Risikoaversion}    &\Leftrightarrow \frac{\delta U}{\delta \sigma} < 0 \\
      \text{Risikoneutralität} &\Leftrightarrow \frac{\delta U}{\delta \sigma} = 0 \\
      \text{Risikofreude}      &\Leftrightarrow \frac{\delta U}{\delta \sigma} > 0 \\
    \end{split}
  \end{equation*}
\textbf{Sicherheitsäquivalent} (CE):
  \begin{equation*}
    \varphi(CE, 0) = \varphi(\mu, \sigma)
  \end{equation*}
\textbf{Risikoprämie (RP):}
  \begin{equation*}
    RP = E[X] - CE
  \end{equation*}
\textbf{Merke:} ($\mu$,$\sigma$)-Prinzip steht nur bei quadratischen und
exponientiellen Nutzenfunktionen im Einklang mit dem Bernulliding

%-------------------------------------------------------------------------------

\section{Investitionsrechnung}

\subsection{Finanzmathematik}

\textbf{Endwert} (aufzinsen)(terminal value):
  \begin{equation*}
    \begin{split}
      EW &= z_1(1+i)^{T-1} + \dots + z_{T-1}(1+i) + Z_T \\
      EW &= \sum^T_{t=1} z_t (1+i)^{T-t}\\
    \end{split}
  \end{equation*}
\textbf{Barwert} (abzinsen)(present value, Gegenwartswert):
  \begin{equation*}
    \begin{split}
      BW &= \frac{z_1}{1+i}+\frac{z_2}{(1+i)^2}+\dots+\frac{z_T}{(1+i)^T}\\
      BW &= \sum^T_{t=\color{red}{\textbf{1}}}\frac{z_t}{(1+i)^t}\\
    \end{split}
  \end{equation*}
\textbf{Kapitalwert} (abzinsen)(net present value‚ Nettogegenwartswert, Nettobarwert)
  \begin{equation*}
    KW = \sum^T_{t=\color{red}{\textbf{0}}}\frac{z_t}{(1+i)^t}
  \end{equation*}
\textbf{Merke:} Bei PV wird die Anfangsauszahlung nicht abgezogen, beim NPV schon.

\subsection{Investitionsrechnung}

Nach(Vor)schüssig bedeutet, dass Zahlungen direkt nach(vor) dem Zins 
stattfinden. D.h. für den EW, dass ein mal weniger(mehr) verzinst wird und für 
den BW, dass ein mal mehr(weniger) abgezinst wird.\\
\textbf{Ewige Rente (Zahlungsbeginn k):}
  \begin{equation*}
    KW = \sum^{\infty}_{t=\color{red}{\textbf{k}}}\frac{A}{(1+i)^t}=\frac{A}{i}(1+i)^{\color{red}{\textbf{1-k}}}
  \end{equation*}
\textbf{Ewige, nachschüssige Rente:}
  \begin{equation*}
    KW = \sum^{\infty}_{t=\color{red}{\textbf{1}}}\frac{A}{(1+i)^t}=\frac{A}{i}
  \end{equation*}
\textbf{Ewige, vorschüssige  Rente:}
  \begin{equation*}
    KW = \sum^{\infty}_{t=\color{red}{\textbf{0}}}\frac{A}{(1+i)^t}=\frac{A}{i}(1+i)^{\color{lightgray}{\textbf{1}}}
  \end{equation*}
\textbf{Konvergenz geometrischer Reihen:}
  \begin{equation*}\label{convergence}
    \begin{split}
    |\delta|<1 \rightarrow \sum^T_{t=0}\delta^t=\frac{1}{1-\delta}\\
    \implies \forall i > 0 : \sum^T_{t=0}\left(\frac{1}{1+i}\right)^t  = \frac{1+i}{i}\\
    \end{split}
  \end{equation*}
\textbf{Annuität}:\\
  \begin{equation*}
    A = BW \cdot \color{cyan}ANF\color{black} = BW \color{cyan}\left(\frac{(1+i)^T \cdot i}{(1+i)^T-1}\right)\\
  \end{equation*}
\textbf{Rentenbarwert}:\\
  \begin{equation*}
    BW = A \cdot \color{cyan}RBF\color{black} = A  \color{cyan}\left(\frac{(1+i)^T-1}{(1+i)^T \cdot i}\right)
  \end{equation*}
\textbf{Rentenendwert}:\\
  \begin{equation*}
    BW = A \cdot \color{cyan}REF\color{black} = A  \color{cyan}\left(\frac{(1+i)^T-1}{i}\right)
  \end{equation*}  
\textbf{BW $\sim$ EW:}
  \begin{equation*}\label{relation}
    \begin{split}
    &EW = BW (1+i)^{T}\\
    \iff &BW = EW (1+i)^{-T} \\
    \end{split}
  \end{equation*}
\textbf{Interner Zins:}\label{internerzins}
  Es sei ($z_0, z_1, \dots, z_T$) ein Zahlungsstrom. Der interne Zins ist eine Zahl $i^*$ , die 
  die folgende Gleichung löst:
  $$0 = z_0 + \frac{z_1}{(1+i^*)^1} + \cdots + \frac{z_T}{(1+i^*)^T} = \sum^T_{t=0}\frac{z_t}{(1+i)^t}$$
  Ersetze dazu $\delta^* = \frac{1}{(1+i)^t}$ und erreche $i^* = \frac{1}{\delta^*}-1$
  
  
  
  
  
  
  
% \textbf{Annuität}:\\
%   Eine zur Zahlungsreihe $z_t$ ($t$ = 0, 1, ..., T) äquivalente Annuität ist eine Reihe 
%   gleich hoher Zahlungen A zu den Zeitpunkten 1, 2, ..., T, deren Kapitalwert 
%   gleich dem der ursprünglichen Zahlungsreihe ist.
%   \begin{equation*}
%     \begin{split}
%       KW &= \sum^T_{t=1}\frac{A}{(1+i)^t}\\
%       KW &= \color{cyan}\underbrace{\frac{A}{i}}_\text{Ewige Rente (k=1)}
%             \color{black}-
%             \color{magenta}\underbrace{\frac{A}{i}\cdot\frac{1}{(1+i)^T}}_\text{Ewige Rente (k=T)}\\
%       KW &= A\color{orange} \underbrace{\left(\frac{1}{i}-\frac{1}{i(1+i)^T}\right)}_\text{Rentenbarwertfaktor}\\
%       KW &= A\color{orange} \underbrace{\left(\frac{(1+i)^T-1}{(1+i)^T \cdot i}\right)}_\text{Rentenbarwertfaktor}\\
%       A  &= KW \underbrace{\left(\frac{(1+i)^T \cdot i}{(1+i)^T-1}\right)}_\text{Annuitätenfaktor}\\
%     \end{split}
%   \end{equation*}
% \begin{center}
%   \begin{tikzpicture}
%     \draw [|-latex,line width=1pt, cyan] (0,1.0) -- node[above] {1, 2, 3, \dots} (8,1.0);
%     \draw [|-latex,line width=1pt, magenta] (5,0.5) -- node[above] {T+1, \dots} (8,0.5);
%     \draw [|-|,line width=1pt, orange] (0,0.0) -- node[above] {1, 2, 3, \dots, T} (5,0.0);
%   \end{tikzpicture}
% \end{center}


  
  

\subsection{Preinreich-Lücke-Theorem}

\textbf{Kongruenzprinzip:}
  \begin{equation*}\label{Kongruenzprinzip}
    \sum^{T}_{t=0} (e_t-a_t) = \sum^{T}_{t=0} (L_t-K_t)
  \end{equation*}
\textbf{Residualgewinn:}
  \begin{equation*}\label{Residualgewinn}
    RG_t=\underbrace{L_t-K_t}_\text{Periodengewinn}-\underbrace{i \cdot KB_{t-1}}_\text{Kapitalbindungszinsen}
  \end{equation*}
\textbf{Preinreich-Lücke-Theorem:}
  \begin{equation*}
    \begin{split}
      \sum^{T}_{t=0} \frac{RG_t}{(1+i)^t} 
        &= \sum^{T}_{t=0} \frac{L_t - K_t - i \cdot KB_{t-1}}{(1+i)^t}\\
        &\equiv \sum^{T}_{t=0} \frac{e_t - a_t}{(1+i)^t}
        = \sum^{T}_{t=0} \frac{cf_t}{(1+i)^t}
        =: KW
    \end{split}
  \end{equation*}

%-------------------------------------------------------------------------------

\section{Capital Asset Pricing Model (CAPM)}

\subsection{Bewertung des Risikos}
  \textbf{Das Beta – Marktrisikosensitivität:}
    \begin{equation*}
      \beta_i = \frac{Cov(r_i ,r_M)}{Var(r_M)}
    \end{equation*}
  \textbf{Marktrisikoprämie:}
    \begin{equation*}
      MRP = \mu(r_M) -r_f
    \end{equation*}
  \textbf{Risikoprämie für ein WP$_i$:}
    \begin{equation*}
      RP_i = \beta_i \cdot MRP
    \end{equation*}

\subsection{Optimale Portfolioallokation und CAPM}

\textbf{Erwartete Rendite eines Portfolios P aus n WPs:}
  \begin{equation*}
    E[r_P] =\mu_P= \sum^{n}_{i=1} a_i \cdot E[r_i]
  \end{equation*}
\textbf{Varianz eines Portfolios P aus n WPs:}
  \begin{equation*}
    \begin{split}
      Var[r_P] &=\sigma_P^2 = E[(r_p-E[r_p])^2]\\
      Var[r_P] &=\sum^{n}_{i=1}\sum^{n}_{j=1}a_i \cdot a_j \cdot Cov[r_i,r_j]\\
      Var[r_{P_{2}}] &\overset{\color{red}{n=2}}{=} a^2_1 \sigma^2_1 + 2a_1a_2 \cdot Cov[r_1,r_2] + a^2_2 \sigma^2_2
    \end{split}
  \end{equation*}
\textbf{Kovarianz zweier Wertpapiere:}
  \begin{equation*}
      Cov[r_i,r_j] = E[(r_i-E[r_i]) \cdot (r_j-E[r_j])]
  \end{equation*}
\textbf{Korrelation zweier Wertpapiere:}
  \begin{equation*}
      Corr[r_i,r_j] = \varrho_{i,j} = \frac{Cov[r_i,r_j]}{\sigma_i \cdot \sigma_j}\in [-1,1]\\
  \end{equation*}
\textbf{Bedingungen für effiziente Portfolios:}
  \begin{enumerate}
    \item Portfolio minimiert bei gegebener erwarteter Rendite die Varianz
    \item Portfolio wird nicht dominiert
  \end{enumerate}
  
\subsubsection{Kombination von Portfolio $M$ mit risikoloser Anlage $f$:}

Investor mischt $M$ mit einem sicheren Wertpapier $f$ mit der Rendite $r_f$ . Anteil von $f$ betrage $(1 - b)$.\\
\textbf{Erwartete Rendite:}
  \begin{equation*}
    \begin{split}
      E[r_K] &= (1-b) \cdot r_f + b \cdot E[r_M]\\
      E[r_K] &= r_f + b \cdot (E[r_M ] - r_f)
    \end{split}
  \end{equation*}
\textbf{Varianz:}
  \begin{equation*}
      Var[r_K] = b^2 \cdot Var[r_M]
  \end{equation*}
\subsubsection{CAPM:}
\textbf{Annahmen:}
  \begin{itemize}
    \item Anleger sind rational und risikoavers
    \item Der Markt ist vollkommen und friktionsfrei
    \begin{itemize}
      \item[$\bullet$] Homogene Erwartungen bezüglich erwarteter Rendite,\\ Volatilität und Korrelationen 
      \item[$\bullet$] Kauf- und Verkaufspreise sind identisch
      \item[$\bullet$] Keine Transaktionskosten
      \item[$\bullet$] Kurze Positionen (short positions) zulässig
      \item[$\bullet$] Keine Arbitragemöglichkeiten
    \end{itemize}
  \end{itemize} 
\textbf{Kapitalmarktlinie:}
  \begin{equation*}
    \mu_{KML} = r_f + \frac{\mu_M - r_f}{\sigma_M}\cdot \sigma_{KML}
  \end{equation*}
\textbf{Capital-Asset-Pricing-Modell:}
  \begin{equation*}
    \begin{split}
      E[r_i] &= r_f +  \underbrace{\underbrace{(E[r_M] - r_f)}_{MRP} \cdot \underbrace{\frac{Cov[r_i,r_M]}{Var[r_M]}}_{\beta_i}}_{RP}\\
    \end{split}
  \end{equation*}

%-------------------------------------------------------------------------------

\section{Kennzahlen und Performancemaße}

\subsection{Relative Kennzahlen und Performancemaße}

\textbf{Return on Investment (ROI):}
  \begin{equation*}
    ROI_t = \frac{\text{Gewinn}_t}{\text{(Rest-)Buchwert des Eigenkapital}_{t-1}} = \frac{cf_t-ab_t}{KB_{t-1}}
  \end{equation*}
\textbf{Cashflow Return on Investment (CFROI):}
  \begin{equation*}
    CFROI_t = \frac{cf_t-ab^{\text{öko}}}{BIB_0}
  \end{equation*}
  \begin{equation*}
    ab^{\text{öko}} = (BIB_0 - BIB^{\text{nicht abschreibar}}_0) \cdot \frac{i}{(1+i)^T-1}
  \end{equation*}
\textbf{Interner Zins: (Siehe \ref{internerzins})}
  
\subsection{Absolute Kennzahlen und Performancemaße}

\textbf{Cash Value Added (CVA):}
  \begin{equation*}
  \begin{split}
    CVA_t &= (CFROI_t - i_t) \cdot BIB_0\\
          &= cf_t-ab^{\text{öko}}- i\cdot BIB_0
  \end{split}
  \end{equation*}
\textbf{Economic Value Added (EVA) bzw.}\\
\textbf{Residualgewinn: (in KLR, vgl. \ref{Residualgewinn})}
  \begin{equation*}
    EVA_t = RG_t = cf_t - ab_t - i\cdot KB_{t-1}
  \end{equation*}
\textbf{Heidgens Tipp zur degresiven Abschreibung:}
  \begin{equation*}
    KB_t = KB_0(1-\delta)^t
  \end{equation*}
\textbf{Residualgewinn (Degresive Abschreibung):}
  \begin{equation*} 
    RG_t = cf_t - KB_0(i+\delta)(1-\delta)^{t-1}
  \end{equation*}

%-------------------------------------------------------------------------------

\section{Investitionssteuerung}

\textbf{Performancemaß Modell allgemein:}
  \begin{equation*}
    PM_{t,i} = \alpha_{t,i} \cdot cf_{t,i} + \beta_{t,i} \cdot b_{0,i}
  \end{equation*}
\textbf{Strukturparameter:}
  \begin{equation*}
    x_{t,i} =
  \end{equation*}
\textbf{Zusammenhang Cashflow Strukturparameter:}
  \begin{equation*}
    cf_{t,i} = x_{t,i} \cdot h_i \color{lightgray}\cdot \epsilon_{t,i}
  \end{equation*}

\subsection{Abschreibung}

\textbf{Schwache Zielkongruenz:}
  \begin{equation*}
    E[KW_i] > 0 \iff \sum^{T_i}_{t=1} E[PM_{t,i}] > 0 \hspace{0.5cm} \forall\,i
  \end{equation*}
\textbf{Umsetzung durch Abschreibungsverfahren:}
  \begin{equation*}
    PM^{AB}_{t,i} = \underbrace{1}_{\alpha} \cdot cf_{t,i} - \underbrace{\frac{1}{T_i+r \cdot\left[1-\frac{t-1}{T_i})\right]}}_{\beta} \cdot b_{0,i}
  \end{equation*}

\subsection{Relatives Beitragsverfahren}

\textbf{Starke Zielkongruenz:}
  \begin{equation*}
    E[KW_i] > 0 \iff E[PM_{t,i}] > 0 \hspace{0.5cm} \forall\, t,i
  \end{equation*}
\textbf{Umsetzung durch relatives Beitragsverfahren:}
  \begin{equation*}
    PM^{RBV}_{t,i} = \underbrace{1}_{\alpha} \cdot cf_{t,i} - \underbrace{\sum^{T_i}_{\tau=1}x_{t,i}(1+r)^{-\tau}}_{\beta} \cdot b_{0,i}
  \end{equation*}

\subsection{Annuitätenverfahren}

\textbf{Perfekte Zielkongruenz:}
  \begin{align*}
    &E[KW_i] > E[KW_j]\\
    \iff & ( T_i = T_j \to E[PM_{t,i}] > E[PM_{t,j}]) \hspace{0.5cm} \forall\, t,i,j,i \neq j
  \end{align*}
\textbf{Umsetzung durch Annuitätenverfahren:}
  \begin{equation*}
        PM^{ANV}_{t,i} = \underbrace{\frac{A_i}{x_{t,i}} \cdot \sum^{T_i}_{\tau=1}x_{t,i}(1+r)^{-\tau}}_{\alpha} \cdot cf_{t,i} - \underbrace{A_i}_{\beta} \cdot b_{0,i}
  \end{equation*}

\subsection{Bonusbanken}

\textbf{Perfekte Zielkongruenz auch bei unterschiedlichen Laufzeiten:}
  \begin{align*}
    &E[KW_i] > E[KW_j]\\
    \iff & E[PM^{BB}_{t,i}] > E[PM^{BB}_{t,j}] \hspace{0.5cm} \forall\, t,i,j,i \neq j
  \end{align*}
\textbf{Umsetzung durch Bonusbanken:}
``Strecke'' die kürzere Laufzeit. Wähle $\delta = \frac{A_j}{A_i}$, wenn $T_i < T_j$, 1 sonst.
  \begin{equation*}
    PM^{BB}_{t,i} = \delta \cdot PM^{ANV}_{t,i} = \delta \cdot A_{i} \cdot KW_i
  \end{equation*}

%-------------------------------------------------------------------------------

\appendix 
\section{Vereinfachende Annahmen}
$I=b_0=cf_0=KB_0=BIB_0$\\
$bcf=ocf=cf$\\
$ab=AB=AfA$  


\end{multicols}
\end{document}


